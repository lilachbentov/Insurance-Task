\documentclass[]{article}
\usepackage{lmodern}
\usepackage{amssymb,amsmath}
\usepackage{ifxetex,ifluatex}
\usepackage{fixltx2e} % provides \textsubscript
\ifnum 0\ifxetex 1\fi\ifluatex 1\fi=0 % if pdftex
  \usepackage[T1]{fontenc}
  \usepackage[utf8]{inputenc}
\else % if luatex or xelatex
  \ifxetex
    \usepackage{mathspec}
  \else
    \usepackage{fontspec}
  \fi
  \defaultfontfeatures{Ligatures=TeX,Scale=MatchLowercase}
\fi
% use upquote if available, for straight quotes in verbatim environments
\IfFileExists{upquote.sty}{\usepackage{upquote}}{}
% use microtype if available
\IfFileExists{microtype.sty}{%
\usepackage{microtype}
\UseMicrotypeSet[protrusion]{basicmath} % disable protrusion for tt fonts
}{}
\usepackage[margin=1in]{geometry}
\usepackage{hyperref}
\hypersetup{unicode=true,
            pdftitle={Lilach Ben Tov - Task},
            pdfborder={0 0 0},
            breaklinks=true}
\urlstyle{same}  % don't use monospace font for urls
\usepackage{color}
\usepackage{fancyvrb}
\newcommand{\VerbBar}{|}
\newcommand{\VERB}{\Verb[commandchars=\\\{\}]}
\DefineVerbatimEnvironment{Highlighting}{Verbatim}{commandchars=\\\{\}}
% Add ',fontsize=\small' for more characters per line
\usepackage{framed}
\definecolor{shadecolor}{RGB}{248,248,248}
\newenvironment{Shaded}{\begin{snugshade}}{\end{snugshade}}
\newcommand{\KeywordTok}[1]{\textcolor[rgb]{0.13,0.29,0.53}{\textbf{#1}}}
\newcommand{\DataTypeTok}[1]{\textcolor[rgb]{0.13,0.29,0.53}{#1}}
\newcommand{\DecValTok}[1]{\textcolor[rgb]{0.00,0.00,0.81}{#1}}
\newcommand{\BaseNTok}[1]{\textcolor[rgb]{0.00,0.00,0.81}{#1}}
\newcommand{\FloatTok}[1]{\textcolor[rgb]{0.00,0.00,0.81}{#1}}
\newcommand{\ConstantTok}[1]{\textcolor[rgb]{0.00,0.00,0.00}{#1}}
\newcommand{\CharTok}[1]{\textcolor[rgb]{0.31,0.60,0.02}{#1}}
\newcommand{\SpecialCharTok}[1]{\textcolor[rgb]{0.00,0.00,0.00}{#1}}
\newcommand{\StringTok}[1]{\textcolor[rgb]{0.31,0.60,0.02}{#1}}
\newcommand{\VerbatimStringTok}[1]{\textcolor[rgb]{0.31,0.60,0.02}{#1}}
\newcommand{\SpecialStringTok}[1]{\textcolor[rgb]{0.31,0.60,0.02}{#1}}
\newcommand{\ImportTok}[1]{#1}
\newcommand{\CommentTok}[1]{\textcolor[rgb]{0.56,0.35,0.01}{\textit{#1}}}
\newcommand{\DocumentationTok}[1]{\textcolor[rgb]{0.56,0.35,0.01}{\textbf{\textit{#1}}}}
\newcommand{\AnnotationTok}[1]{\textcolor[rgb]{0.56,0.35,0.01}{\textbf{\textit{#1}}}}
\newcommand{\CommentVarTok}[1]{\textcolor[rgb]{0.56,0.35,0.01}{\textbf{\textit{#1}}}}
\newcommand{\OtherTok}[1]{\textcolor[rgb]{0.56,0.35,0.01}{#1}}
\newcommand{\FunctionTok}[1]{\textcolor[rgb]{0.00,0.00,0.00}{#1}}
\newcommand{\VariableTok}[1]{\textcolor[rgb]{0.00,0.00,0.00}{#1}}
\newcommand{\ControlFlowTok}[1]{\textcolor[rgb]{0.13,0.29,0.53}{\textbf{#1}}}
\newcommand{\OperatorTok}[1]{\textcolor[rgb]{0.81,0.36,0.00}{\textbf{#1}}}
\newcommand{\BuiltInTok}[1]{#1}
\newcommand{\ExtensionTok}[1]{#1}
\newcommand{\PreprocessorTok}[1]{\textcolor[rgb]{0.56,0.35,0.01}{\textit{#1}}}
\newcommand{\AttributeTok}[1]{\textcolor[rgb]{0.77,0.63,0.00}{#1}}
\newcommand{\RegionMarkerTok}[1]{#1}
\newcommand{\InformationTok}[1]{\textcolor[rgb]{0.56,0.35,0.01}{\textbf{\textit{#1}}}}
\newcommand{\WarningTok}[1]{\textcolor[rgb]{0.56,0.35,0.01}{\textbf{\textit{#1}}}}
\newcommand{\AlertTok}[1]{\textcolor[rgb]{0.94,0.16,0.16}{#1}}
\newcommand{\ErrorTok}[1]{\textcolor[rgb]{0.64,0.00,0.00}{\textbf{#1}}}
\newcommand{\NormalTok}[1]{#1}
\usepackage{graphicx,grffile}
\makeatletter
\def\maxwidth{\ifdim\Gin@nat@width>\linewidth\linewidth\else\Gin@nat@width\fi}
\def\maxheight{\ifdim\Gin@nat@height>\textheight\textheight\else\Gin@nat@height\fi}
\makeatother
% Scale images if necessary, so that they will not overflow the page
% margins by default, and it is still possible to overwrite the defaults
% using explicit options in \includegraphics[width, height, ...]{}
\setkeys{Gin}{width=\maxwidth,height=\maxheight,keepaspectratio}
\IfFileExists{parskip.sty}{%
\usepackage{parskip}
}{% else
\setlength{\parindent}{0pt}
\setlength{\parskip}{6pt plus 2pt minus 1pt}
}
\setlength{\emergencystretch}{3em}  % prevent overfull lines
\providecommand{\tightlist}{%
  \setlength{\itemsep}{0pt}\setlength{\parskip}{0pt}}
\setcounter{secnumdepth}{0}
% Redefines (sub)paragraphs to behave more like sections
\ifx\paragraph\undefined\else
\let\oldparagraph\paragraph
\renewcommand{\paragraph}[1]{\oldparagraph{#1}\mbox{}}
\fi
\ifx\subparagraph\undefined\else
\let\oldsubparagraph\subparagraph
\renewcommand{\subparagraph}[1]{\oldsubparagraph{#1}\mbox{}}
\fi

%%% Use protect on footnotes to avoid problems with footnotes in titles
\let\rmarkdownfootnote\footnote%
\def\footnote{\protect\rmarkdownfootnote}

%%% Change title format to be more compact
\usepackage{titling}

% Create subtitle command for use in maketitle
\newcommand{\subtitle}[1]{
  \posttitle{
    \begin{center}\large#1\end{center}
    }
}

\setlength{\droptitle}{-2em}

  \title{Lilach Ben Tov - Task}
    \pretitle{\vspace{\droptitle}\centering\huge}
  \posttitle{\par}
    \author{}
    \preauthor{}\postauthor{}
    \date{}
    \predate{}\postdate{}
  

\begin{document}
\maketitle

Install and upload packages:

Importing the data sets:

\begin{Shaded}
\begin{Highlighting}[]
\NormalTok{Exposure<-}\KeywordTok{read.csv}\NormalTok{(}\DataTypeTok{file=}\StringTok{"files/Exposure_file2.csv"}\NormalTok{,}\DataTypeTok{header=}\OtherTok{TRUE}\NormalTok{, }\DataTypeTok{sep=}\StringTok{","}\NormalTok{)}
\NormalTok{Claims<-}\KeywordTok{read.csv}\NormalTok{(}\DataTypeTok{file=}\StringTok{"files/Claims_file.csv"}\NormalTok{,}\DataTypeTok{header=}\OtherTok{TRUE}\NormalTok{, }\DataTypeTok{sep=}\StringTok{","}\NormalTok{, )}
\end{Highlighting}
\end{Shaded}

\subsection{Step 1 - Data
preprocessing}\label{step-1---data-preprocessing}

Checking for missing values and validating data.

\subsubsection{Preprocessing Exposure
Table}\label{preprocessing-exposure-table}

\begin{Shaded}
\begin{Highlighting}[]
\KeywordTok{class}\NormalTok{(Exposure)}
\end{Highlighting}
\end{Shaded}

\begin{verbatim}
## [1] "data.frame"
\end{verbatim}

\begin{Shaded}
\begin{Highlighting}[]
\KeywordTok{dim}\NormalTok{(Exposure)}
\end{Highlighting}
\end{Shaded}

\begin{verbatim}
## [1] 250000      7
\end{verbatim}

\begin{Shaded}
\begin{Highlighting}[]
\KeywordTok{summary}\NormalTok{(Exposure)}
\end{Highlighting}
\end{Shaded}

\begin{verbatim}
##        ID           SMOKER                    OCCUPATION    
##  Min.   :     1   Mode :logical   BLUE_COLLAR      : 74934  
##  1st Qu.: 62501   FALSE:174994    HEAVY_BLUE_COLLAR: 25095  
##  Median :125000   TRUE :75006     WHITE_COLLAR     :149971  
##  Mean   :125000                                             
##  3rd Qu.:187500                                             
##  Max.   :250000                                             
##                                                             
##          DOB          POLICY_START_DATE    POLICY_END_DATE  
##  8/7/1982  :    42   10/29/2011:    97   10/1/2018 :    39  
##  11/19/1979:    40   7/25/2009 :    97   7/23/2018 :    38  
##  5/27/1982 :    40   9/19/2011 :    97   4/28/2017 :    37  
##  11/1/1980 :    39   11/16/2010:    96   1/10/2017 :    36  
##  3/9/1969  :    38   6/29/2017 :    96   10/20/2018:    36  
##  9/19/1989 :    38   3/6/2012  :    93   (Other)   : 44433  
##  (Other)   :249763   (Other)   :249424   NA's      :205381  
##  TOTAL_PREMIUM_PAID_TO_DATE
##  Min.   :  0.0             
##  1st Qu.:143.4             
##  Median :302.0             
##  Mean   :322.0             
##  3rd Qu.:488.0             
##  Max.   :730.4             
## 
\end{verbatim}

\begin{Shaded}
\begin{Highlighting}[]
\KeywordTok{sapply}\NormalTok{(Exposure,class)}
\end{Highlighting}
\end{Shaded}

\begin{verbatim}
##                         ID                     SMOKER 
##                  "integer"                  "logical" 
##                 OCCUPATION                        DOB 
##                   "factor"                   "factor" 
##          POLICY_START_DATE            POLICY_END_DATE 
##                   "factor"                   "factor" 
## TOTAL_PREMIUM_PAID_TO_DATE 
##                  "numeric"
\end{verbatim}

By looking at the summary report I validated that there are no
casesensitive mismatch and neither missing data. In addtion, I found
that the dates columns are defined as \emph{factors} and not as
\emph{date} type, therefore I defined them as \emph{date} tpye by using
\texttt{as.Date} function:

\begin{Shaded}
\begin{Highlighting}[]
\NormalTok{Exposure[,}\DecValTok{4}\NormalTok{]<-}\KeywordTok{as.Date}\NormalTok{(Exposure[,}\DecValTok{4}\NormalTok{],}\DataTypeTok{format=}\KeywordTok{c}\NormalTok{(}\StringTok{"%m/%d/%Y"}\NormalTok{))}
\NormalTok{Exposure[,}\DecValTok{5}\NormalTok{]<-}\KeywordTok{as.Date}\NormalTok{(Exposure[,}\DecValTok{5}\NormalTok{],}\DataTypeTok{format=}\KeywordTok{c}\NormalTok{(}\StringTok{"%m/%d/%Y"}\NormalTok{))}
\NormalTok{Exposure[,}\DecValTok{6}\NormalTok{]<-}\KeywordTok{as.Date}\NormalTok{(Exposure[,}\DecValTok{6}\NormalTok{],}\DataTypeTok{format=}\KeywordTok{c}\NormalTok{(}\StringTok{"%m/%d/%Y"}\NormalTok{))}
\end{Highlighting}
\end{Shaded}

Validate the change

\begin{Shaded}
\begin{Highlighting}[]
\KeywordTok{summary}\NormalTok{(Exposure)}
\end{Highlighting}
\end{Shaded}

\begin{verbatim}
##        ID           SMOKER                    OCCUPATION    
##  Min.   :     1   Mode :logical   BLUE_COLLAR      : 74934  
##  1st Qu.: 62501   FALSE:174994    HEAVY_BLUE_COLLAR: 25095  
##  Median :125000   TRUE :75006     WHITE_COLLAR     :149971  
##  Mean   :125000                                             
##  3rd Qu.:187500                                             
##  Max.   :250000                                             
##                                                             
##       DOB             POLICY_START_DATE    POLICY_END_DATE     
##  Min.   :1958-12-27   Min.   :2008-11-30   Min.   :2009-02-20  
##  1st Qu.:1971-12-03   1st Qu.:2011-06-02   1st Qu.:2013-12-31  
##  Median :1979-11-27   Median :2013-11-29   Median :2015-12-24  
##  Mean   :1979-11-28   Mean   :2013-11-29   Mean   :2015-08-06  
##  3rd Qu.:1987-11-28   3rd Qu.:2016-06-01   3rd Qu.:2017-07-12  
##  Max.   :2000-11-19   Max.   :2018-11-30   Max.   :2018-10-31  
##                                            NA's   :205381      
##  TOTAL_PREMIUM_PAID_TO_DATE
##  Min.   :  0.0             
##  1st Qu.:143.4             
##  Median :302.0             
##  Mean   :322.0             
##  3rd Qu.:488.0             
##  Max.   :730.4             
## 
\end{verbatim}

\begin{Shaded}
\begin{Highlighting}[]
\KeywordTok{sapply}\NormalTok{(Exposure,class)}
\end{Highlighting}
\end{Shaded}

\begin{verbatim}
##                         ID                     SMOKER 
##                  "integer"                  "logical" 
##                 OCCUPATION                        DOB 
##                   "factor"                     "Date" 
##          POLICY_START_DATE            POLICY_END_DATE 
##                     "Date"                     "Date" 
## TOTAL_PREMIUM_PAID_TO_DATE 
##                  "numeric"
\end{verbatim}

I wrote a utility function that iterates over a data frame and returns
columns with missing data

\begin{Shaded}
\begin{Highlighting}[]
\NormalTok{NAfunction<-}\ControlFlowTok{function}\NormalTok{(x)\{}
\NormalTok{  dim2<-}\KeywordTok{dim}\NormalTok{(x)[}\DecValTok{2}\NormalTok{]}
\NormalTok{  vec<-}\KeywordTok{rep.int}\NormalTok{(}\DecValTok{0}\NormalTok{,}\KeywordTok{dim}\NormalTok{(x)[}\DecValTok{2}\NormalTok{])}
  \ControlFlowTok{for}\NormalTok{ (i }\ControlFlowTok{in} \DecValTok{1}\OperatorTok{:}\NormalTok{dim2)\{}
    \ControlFlowTok{if}\NormalTok{ (}\KeywordTok{anyNA}\NormalTok{(Exposure[,i])}\OperatorTok{==}\OtherTok{TRUE}\NormalTok{)\{}
\NormalTok{      vec[i]<-}\DecValTok{1}
\NormalTok{      i<-i}\OperatorTok{+}\DecValTok{1}
\NormalTok{    \}}\ControlFlowTok{else}
\NormalTok{      i<-i}\OperatorTok{+}\DecValTok{1}

\NormalTok{  \}}
  \KeywordTok{return}\NormalTok{(}\KeywordTok{colnames}\NormalTok{(x[}\KeywordTok{which}\NormalTok{(vec}\OperatorTok{>}\DecValTok{0}\NormalTok{)]))}
\NormalTok{\}}
\end{Highlighting}
\end{Shaded}

Running the function on the data frame:

\begin{Shaded}
\begin{Highlighting}[]
\KeywordTok{NAfunction}\NormalTok{(Exposure)}
\end{Highlighting}
\end{Shaded}

\begin{verbatim}
## [1] "POLICY_END_DATE"
\end{verbatim}

As expected, only the sixth column has NA values

Removal of duplicate rows (if there any):

\begin{Shaded}
\begin{Highlighting}[]
\NormalTok{CleanExposure<-}\KeywordTok{unique}\NormalTok{(Exposure)}
\end{Highlighting}
\end{Shaded}

\subsubsection{Preprocessing Claims
Table}\label{preprocessing-claims-table}

\begin{Shaded}
\begin{Highlighting}[]
\KeywordTok{class}\NormalTok{(Claims)}
\end{Highlighting}
\end{Shaded}

\begin{verbatim}
## [1] "data.frame"
\end{verbatim}

\begin{Shaded}
\begin{Highlighting}[]
\KeywordTok{dim}\NormalTok{(Claims)}
\end{Highlighting}
\end{Shaded}

\begin{verbatim}
## [1] 11589     3
\end{verbatim}

\begin{Shaded}
\begin{Highlighting}[]
\KeywordTok{summary}\NormalTok{(Claims)}
\end{Highlighting}
\end{Shaded}

\begin{verbatim}
##        ID           CLAIM_SIZE         CLAIM_DATE   
##  Min.   :     6   Min.   : 3608   20/03/2018:   18  
##  1st Qu.: 62502   1st Qu.: 4211   23/04/2018:   18  
##  Median :125674   Median : 4784   22/05/2017:   17  
##  Mean   :124708   Mean   : 7776   25/04/2017:   17  
##  3rd Qu.:186548   3rd Qu.: 5390   31/10/2018:   17  
##  Max.   :249975   Max.   :21648   02/06/2018:   16  
##                                   (Other)   :11486
\end{verbatim}

\begin{Shaded}
\begin{Highlighting}[]
\KeywordTok{sapply}\NormalTok{(Claims,class)}
\end{Highlighting}
\end{Shaded}

\begin{verbatim}
##         ID CLAIM_SIZE CLAIM_DATE 
##  "integer"  "numeric"   "factor"
\end{verbatim}

Defining the third column class as type \emph{date}

\begin{Shaded}
\begin{Highlighting}[]
\NormalTok{Claims[,}\DecValTok{3}\NormalTok{]<-}\KeywordTok{as.Date}\NormalTok{(Claims[,}\DecValTok{3}\NormalTok{],}\DataTypeTok{format=}\KeywordTok{c}\NormalTok{(}\StringTok{"%d/%m/%Y"}\NormalTok{))}
\end{Highlighting}
\end{Shaded}

Validate the change

\begin{Shaded}
\begin{Highlighting}[]
\KeywordTok{summary}\NormalTok{(Claims)}
\end{Highlighting}
\end{Shaded}

\begin{verbatim}
##        ID           CLAIM_SIZE      CLAIM_DATE        
##  Min.   :     6   Min.   : 3608   Min.   :2009-06-19  
##  1st Qu.: 62502   1st Qu.: 4211   1st Qu.:2014-11-01  
##  Median :125674   Median : 4784   Median :2016-07-18  
##  Mean   :124708   Mean   : 7776   Mean   :2016-02-17  
##  3rd Qu.:186548   3rd Qu.: 5390   3rd Qu.:2017-10-20  
##  Max.   :249975   Max.   :21648   Max.   :2018-10-31
\end{verbatim}

\begin{Shaded}
\begin{Highlighting}[]
\KeywordTok{sapply}\NormalTok{(Claims,class)}
\end{Highlighting}
\end{Shaded}

\begin{verbatim}
##         ID CLAIM_SIZE CLAIM_DATE 
##  "integer"  "numeric"     "Date"
\end{verbatim}

Looking for NA columns:

\begin{Shaded}
\begin{Highlighting}[]
\KeywordTok{NAfunction}\NormalTok{(Claims)}
\end{Highlighting}
\end{Shaded}

\begin{verbatim}
## character(0)
\end{verbatim}

As expected, no missing data

Removal of duplicate rows (if there any):

\begin{Shaded}
\begin{Highlighting}[]
\NormalTok{CleanClaims<-}\KeywordTok{unique}\NormalTok{(Claims)}
\end{Highlighting}
\end{Shaded}

\subsection{\texorpdfstring{Adding new column \textbf{age}
-\textgreater{} calculate the user age based on the date of
birth}{Adding new column age -\textgreater{} calculate the user age based on the date of birth}}\label{adding-new-column-age---calculate-the-user-age-based-on-the-date-of-birth}

\begin{Shaded}
\begin{Highlighting}[]
\NormalTok{CleanExposure[,}\DecValTok{8}\NormalTok{]<-}\KeywordTok{age_calc}\NormalTok{(CleanExposure}\OperatorTok{$}\NormalTok{DOB,}\DataTypeTok{units=}\StringTok{"years"}\NormalTok{,}\DataTypeTok{precise=}\OtherTok{FALSE}\NormalTok{)}
\KeywordTok{colnames}\NormalTok{(CleanExposure)[}\DecValTok{8}\NormalTok{]<-}\StringTok{"Age"}
\end{Highlighting}
\end{Shaded}

\subsection{\texorpdfstring{adding new column \textbf{duration of being
insured}: -\textgreater{} calculates the duration of being
insured}{adding new column duration of being insured: -\textgreater{} calculates the duration of being insured}}\label{adding-new-column-duration-of-being-insured---calculates-the-duration-of-being-insured}

\begin{Shaded}
\begin{Highlighting}[]
\NormalTok{CleanExposure[}\KeywordTok{is.na}\NormalTok{(CleanExposure}\OperatorTok{$}\NormalTok{POLICY_END_DATE),}\DecValTok{6}\NormalTok{]<-}\KeywordTok{as.Date}\NormalTok{(}\StringTok{'2018-12-01'}\NormalTok{)}
\NormalTok{CleanExposure[,}\DecValTok{9}\NormalTok{]<-}\KeywordTok{age_calc}\NormalTok{(CleanExposure}\OperatorTok{$}\NormalTok{POLICY_START_DATE,}\DataTypeTok{enddate=}\KeywordTok{as.Date}\NormalTok{(CleanExposure}\OperatorTok{$}\NormalTok{POLICY_END_DATE),}\DataTypeTok{units=}\StringTok{"years"}\NormalTok{, }\DataTypeTok{precise=}\OtherTok{FALSE}\NormalTok{)}
\KeywordTok{colnames}\NormalTok{(CleanExposure)[}\DecValTok{9}\NormalTok{]<-}\StringTok{"Duration_by_years"}

\NormalTok{CleanExposure[,}\DecValTok{6}\NormalTok{]<-}\KeywordTok{as.Date}\NormalTok{(Exposure[,}\DecValTok{6}\NormalTok{],}\DataTypeTok{format=}\KeywordTok{c}\NormalTok{(}\StringTok{"%d/%m/%Y"}\NormalTok{))}
\NormalTok{CleanExposure[,}\DecValTok{9}\NormalTok{]<-}\KeywordTok{as.integer}\NormalTok{(CleanExposure[,}\DecValTok{9}\NormalTok{])}
\end{Highlighting}
\end{Shaded}

\subsection{Run SQL Queries}\label{run-sql-queries}

Create a table \textbf{SumClaims} that caluclates the number of claims
and claim size per user id.

Createing a table \textbf{SumData} that joins \textbf{SumClaims} table
and the \textbf{Exposure} table by user id and calculates the profit by
user id

\begin{Shaded}
\begin{Highlighting}[]
\NormalTok{SumClaims<-}\KeywordTok{sqldf}\NormalTok{(}\StringTok{"select ID, sum(CLAIM_SIZE) AS CLAIM_SIZE, COUNT(*) AS NUMBER_OF_CLAIMS}
\StringTok{ FROM CleanClaims group by 1 "}\NormalTok{)}

\NormalTok{SumData<-}\KeywordTok{sqldf}\NormalTok{(}\StringTok{"select E.*, (C.CLAIM_SIZE), (C.NUMBER_OF_CLAIMS),}
\StringTok{(case when C.CLAIM_SIZE is null then E.TOTAL_PREMIUM_PAID_TO_DATE else E.TOTAL_PREMIUM_PAID_TO_DATE-C.CLAIM_SIZE END) AS PROFIT}
\StringTok{FROM CleanExposure E left join SumClaims C on E.ID=C.ID "}\NormalTok{)}
\end{Highlighting}
\end{Shaded}

\subsection{\texorpdfstring{Adding new column \textbf{Margin}
-\textgreater{} profit /
revenue}{Adding new column Margin -\textgreater{} profit / revenue}}\label{adding-new-column-margin---profit-revenue}

\begin{Shaded}
\begin{Highlighting}[]
\NormalTok{SumData[,}\DecValTok{13}\NormalTok{]<-SumData[,}\DecValTok{12}\NormalTok{]}\OperatorTok{/}\NormalTok{SumData[,}\DecValTok{7}\NormalTok{]}
\KeywordTok{colnames}\NormalTok{(SumData)[}\DecValTok{13}\NormalTok{]<-}\StringTok{"Margin"}
\end{Highlighting}
\end{Shaded}

Since there is no revenue for users that started the insurance policy at
the 30-11-2018 I removed those users

\begin{Shaded}
\begin{Highlighting}[]
\NormalTok{SumData<-SumData[}\KeywordTok{c}\NormalTok{(SumData}\OperatorTok{$}\NormalTok{TOTAL_PREMIUM_PAID_TO_DATE}\OperatorTok{>}\DecValTok{0}\NormalTok{),]}
\end{Highlighting}
\end{Shaded}

Setting users that didn't submit a claim (null value) to 0

\begin{Shaded}
\begin{Highlighting}[]
\NormalTok{SumData}\OperatorTok{$}\NormalTok{NUMBER_OF_CLAIMS =}\StringTok{ }\KeywordTok{ifelse}\NormalTok{(}\KeywordTok{is.na}\NormalTok{(SumData}\OperatorTok{$}\NormalTok{NUMBER_OF_CLAIMS), }\DecValTok{0}\NormalTok{, SumData}\OperatorTok{$}\NormalTok{NUMBER_OF_CLAIMS)}

\NormalTok{SumData}\OperatorTok{$}\NormalTok{CLAIM_SIZE =}\StringTok{ }\KeywordTok{ifelse}\NormalTok{(}\KeywordTok{is.na}\NormalTok{(SumData}\OperatorTok{$}\NormalTok{CLAIM_SIZE), }\DecValTok{0}\NormalTok{, SumData}\OperatorTok{$}\NormalTok{CLAIM_SIZE)}
\end{Highlighting}
\end{Shaded}

\subsection{Step 2 - Build Regression
model}\label{step-2---build-regression-model}

\subsubsection{Build multiple regression
model:}\label{build-multiple-regression-model}

Margin = a + b1xSMOKER + b2xOCCUPATION + b3xAge + b4xDuration\_by\_days
\#\#\# In order to measure the most profitable/costy customers for the
company and to remove the difference between old and new customers over
time I chose to use Margin (profit / revenw) as my dependent value

\begin{Shaded}
\begin{Highlighting}[]
\NormalTok{dataset<-SumData[,}\KeywordTok{c}\NormalTok{(}\DecValTok{2}\NormalTok{,}\DecValTok{3}\NormalTok{,}\DecValTok{8}\NormalTok{,}\DecValTok{9}\NormalTok{,}\DecValTok{13}\NormalTok{)]}
\end{Highlighting}
\end{Shaded}

\subsubsection{Encoding categorical
data}\label{encoding-categorical-data}

\begin{Shaded}
\begin{Highlighting}[]
\NormalTok{dataset<-SumData[,}\KeywordTok{c}\NormalTok{(}\DecValTok{2}\NormalTok{,}\DecValTok{3}\NormalTok{,}\DecValTok{8}\NormalTok{,}\DecValTok{9}\NormalTok{,}\DecValTok{13}\NormalTok{)]}
\NormalTok{dataset}\OperatorTok{$}\NormalTok{SMOKER =}\StringTok{ }\KeywordTok{factor}\NormalTok{(dataset}\OperatorTok{$}\NormalTok{SMOKER,}
                       \DataTypeTok{levels =} \KeywordTok{c}\NormalTok{(}\StringTok{'FALSE'}\NormalTok{,}\StringTok{'TRUE'}\NormalTok{),}
                       \DataTypeTok{labels =} \KeywordTok{c}\NormalTok{(}\DecValTok{0}\NormalTok{,}\DecValTok{1}\NormalTok{))}
\NormalTok{dataset}\OperatorTok{$}\NormalTok{OCCUPATION =}\StringTok{ }\KeywordTok{factor}\NormalTok{(dataset}\OperatorTok{$}\NormalTok{OCCUPATION,}
                        \DataTypeTok{levels =} \KeywordTok{c}\NormalTok{(}\StringTok{'BLUE_COLLAR'}\NormalTok{,}\StringTok{'HEAVY_BLUE_COLLAR'}\NormalTok{, }\StringTok{'WHITE_COLLAR'}\NormalTok{ ),}
                        \DataTypeTok{labels =} \KeywordTok{c}\NormalTok{(}\DecValTok{1}\NormalTok{,}\DecValTok{2}\NormalTok{,}\DecValTok{3}\NormalTok{))}
\end{Highlighting}
\end{Shaded}

\subsubsection{Feature Scaling (Age and
duration)}\label{feature-scaling-age-and-duration}

\begin{Shaded}
\begin{Highlighting}[]
\NormalTok{dataset[,}\DecValTok{3}\OperatorTok{:}\DecValTok{4}\NormalTok{] =}\StringTok{ }\KeywordTok{scale}\NormalTok{(dataset[,}\DecValTok{3}\OperatorTok{:}\DecValTok{4}\NormalTok{])}
\end{Highlighting}
\end{Shaded}

\subsubsection{Fitting Multiple Linear
Regression}\label{fitting-multiple-linear-regression}

\begin{Shaded}
\begin{Highlighting}[]
\NormalTok{regressor =}\StringTok{ }\KeywordTok{lm}\NormalTok{(}\DataTypeTok{formula =}\NormalTok{ Margin }\OperatorTok{~}\StringTok{ }\NormalTok{. ,}
               \DataTypeTok{data =}\NormalTok{ dataset)}

\KeywordTok{summary}\NormalTok{(regressor)}
\end{Highlighting}
\end{Shaded}

\begin{verbatim}
## 
## Call:
## lm(formula = Margin ~ ., data = dataset)
## 
## Residuals:
##      Min       1Q   Median       3Q      Max 
## -1208.39     0.20     1.08     1.96     4.72 
## 
## Coefficients:
##                   Estimate Std. Error t value Pr(>|t|)    
## (Intercept)       -1.13734    0.04404 -25.823   <2e-16 ***
## SMOKER1           -0.02353    0.04958  -0.475    0.635    
## OCCUPATION2       -0.07714    0.08285  -0.931    0.352    
## OCCUPATION3        1.60635    0.05082  31.608   <2e-16 ***
## Age               -0.73359    0.02348 -31.244   <2e-16 ***
## Duration_by_years  0.70092    0.02348  29.853   <2e-16 ***
## ---
## Signif. codes:  0 '***' 0.001 '**' 0.01 '*' 0.05 '.' 0.1 ' ' 1
## 
## Residual standard error: 11.36 on 249923 degrees of freedom
## Multiple R-squared:  0.01075,    Adjusted R-squared:  0.01073 
## F-statistic: 542.9 on 5 and 249923 DF,  p-value: < 2.2e-16
\end{verbatim}

\subsubsection{Analysis conculsations}\label{analysis-conculsations}

From the analysis we can clearly see that WHITE\_COLLAR \&
Duration\_by\_days are more profitable for the insurance company. Age
has a significant negative significant impact on the company margin.

\subsection{Step 3 - Based on the analysis checking each parameter
individually}\label{step-3---based-on-the-analysis-checking-each-parameter-individually}

\subsubsection{Checking the OCCUPATION
parameter:}\label{checking-the-occupation-parameter}

\begin{Shaded}
\begin{Highlighting}[]
\NormalTok{SumData<-}\KeywordTok{as.data.table}\NormalTok{(SumData)}

\NormalTok{Occupation_table<-}\KeywordTok{sqldf}\NormalTok{(}\StringTok{"select OCCUPATION, sum(Margin) as Sum_Margin, avg(Margin) as mean_Margin, (avg(Margin*Margin)-avg(Margin)*Avg(Margin)) as Variance}
\StringTok{                        from SumData group by 1 "}\NormalTok{)}

\NormalTok{Occupation_table}
\end{Highlighting}
\end{Shaded}

\begin{verbatim}
##          OCCUPATION Sum_Margin mean_Margin  Variance
## 1       BLUE_COLLAR  -85527.24  -1.1416876 348.91768
## 2 HEAVY_BLUE_COLLAR  -30562.79  -1.2181750 107.98160
## 3      WHITE_COLLAR   68970.87   0.4600297  23.93252
\end{verbatim}

\begin{Shaded}
\begin{Highlighting}[]
\KeywordTok{ggplot}\NormalTok{(SumData, }\KeywordTok{aes}\NormalTok{(}\DataTypeTok{x =}\NormalTok{ OCCUPATION, }\DataTypeTok{y =}\NormalTok{ Margin)) }\OperatorTok{+}
\StringTok{  }\KeywordTok{geom_boxplot}\NormalTok{(}\DataTypeTok{fill =} \StringTok{"grey80"}\NormalTok{, }\DataTypeTok{colour =} \StringTok{"black"}\NormalTok{) }\OperatorTok{+}
\StringTok{  }\KeywordTok{scale_x_discrete}\NormalTok{() }\OperatorTok{+}\StringTok{ }\KeywordTok{xlab}\NormalTok{(}\StringTok{"OCCUPATION Group"}\NormalTok{) }\OperatorTok{+}
\StringTok{  }\KeywordTok{ylab}\NormalTok{(}\StringTok{"Margin"}\NormalTok{)}
\end{Highlighting}
\end{Shaded}

\includegraphics{Task_files/figure-latex/unnamed-chunk-24-1.pdf}

\subsubsection{Analysis conculsations}\label{analysis-conculsations-1}

The mean margin is much higher for WHITE\_COLLAR and Variance
significantly small (imply for less cost claims).

\subsubsection{Checking the age
parameter:}\label{checking-the-age-parameter}

\begin{Shaded}
\begin{Highlighting}[]
\NormalTok{Age_table<-}\KeywordTok{sqldf}\NormalTok{(}\StringTok{"select Age, sum(Margin) as Sum_Margin, avg(Margin) as mean_Margin, (avg(Margin*Margin)-avg(Margin)*Avg(Margin)) as Variance,}
\StringTok{                       sum(NUMBER_OF_CLAIMS) as NUMBER_OF_CLAIMS, count(*), sum(NUMBER_OF_CLAIMS)/count(*)*100 as percentage from SumData group by 1 "}\NormalTok{)}

\KeywordTok{ggplot}\NormalTok{() }\OperatorTok{+}
\StringTok{  }\KeywordTok{geom_point}\NormalTok{(}\KeywordTok{aes}\NormalTok{(}\DataTypeTok{x =}\NormalTok{ Age_table}\OperatorTok{$}\NormalTok{Age, }\DataTypeTok{y =}\NormalTok{ Age_table}\OperatorTok{$}\NormalTok{mean_Margin),}
             \DataTypeTok{colour =} \StringTok{'red'}\NormalTok{) }\OperatorTok{+}
\StringTok{  }\KeywordTok{geom_line}\NormalTok{(}\KeywordTok{aes}\NormalTok{(}\DataTypeTok{x =}\NormalTok{ Age_table}\OperatorTok{$}\NormalTok{Age, }\DataTypeTok{y =}\NormalTok{ Age_table}\OperatorTok{$}\NormalTok{mean_Margin),}
            \DataTypeTok{colour =} \StringTok{'blue'}\NormalTok{) }\OperatorTok{+}
\StringTok{  }\KeywordTok{ggtitle}\NormalTok{(}\StringTok{'Mean Margin vs Age'}\NormalTok{) }\OperatorTok{+}
\StringTok{  }\KeywordTok{xlab}\NormalTok{(}\StringTok{'Age'}\NormalTok{) }\OperatorTok{+}
\StringTok{  }\KeywordTok{ylab}\NormalTok{(}\StringTok{'Mean Margin'}\NormalTok{)}
\end{Highlighting}
\end{Shaded}

\includegraphics{Task_files/figure-latex/unnamed-chunk-25-1.pdf}

\begin{Shaded}
\begin{Highlighting}[]
\KeywordTok{ggplot}\NormalTok{() }\OperatorTok{+}
\StringTok{  }\KeywordTok{geom_point}\NormalTok{(}\KeywordTok{aes}\NormalTok{(}\DataTypeTok{x =}\NormalTok{ Age_table}\OperatorTok{$}\NormalTok{Age, }\DataTypeTok{y =}\NormalTok{ Age_table}\OperatorTok{$}\NormalTok{percentage),}
             \DataTypeTok{colour =} \StringTok{'red'}\NormalTok{) }\OperatorTok{+}
\StringTok{  }\KeywordTok{geom_line}\NormalTok{(}\KeywordTok{aes}\NormalTok{(}\DataTypeTok{x =}\NormalTok{ Age_table}\OperatorTok{$}\NormalTok{Age, }\DataTypeTok{y =}\NormalTok{ Age_table}\OperatorTok{$}\NormalTok{percentage),}
            \DataTypeTok{colour =} \StringTok{'blue'}\NormalTok{) }\OperatorTok{+}
\StringTok{  }\KeywordTok{ggtitle}\NormalTok{(}\StringTok{'Percentage of claims vs Age'}\NormalTok{) }\OperatorTok{+}
\StringTok{  }\KeywordTok{xlab}\NormalTok{(}\StringTok{'Age'}\NormalTok{) }\OperatorTok{+}
\StringTok{  }\KeywordTok{ylab}\NormalTok{(}\StringTok{'Percentage of claim'}\NormalTok{)}
\end{Highlighting}
\end{Shaded}

\includegraphics{Task_files/figure-latex/unnamed-chunk-25-2.pdf}

\subsubsection{Analysis conculsations}\label{analysis-conculsations-2}

There is a decrease in the margin starting at the age of 50 up to the
age of 60. We can see a jump above the age of 60 but that might be
related to the fact that we had only 4 customers in our data. There is a
significant increase in the percentage of claims from age 50 and above.

\subsubsection{Checking the duration:}\label{checking-the-duration}

\begin{Shaded}
\begin{Highlighting}[]
\NormalTok{Duration_table<-}\KeywordTok{sqldf}\NormalTok{(}\StringTok{"select Duration_by_years, sum(Margin) as Sum_Margin, }
\StringTok{avg(Margin) as mean_Margin, }
\StringTok{(avg(Margin*Margin)-avg(Margin)*Avg(Margin)) as Variance,}
\StringTok{sum(NUMBER_OF_CLAIMS) as NUMBER_OF_CLAIMS, count(*), sum(NUMBER_OF_CLAIMS)/count(*)*100 as percentage}
\StringTok{                 from SumData group by 1 "}\NormalTok{)}


\KeywordTok{ggplot}\NormalTok{() }\OperatorTok{+}
\StringTok{  }\KeywordTok{geom_point}\NormalTok{(}\KeywordTok{aes}\NormalTok{(}\DataTypeTok{x =}\NormalTok{ Duration_table}\OperatorTok{$}\NormalTok{Duration_by_years, }\DataTypeTok{y =}\NormalTok{ Duration_table}\OperatorTok{$}\NormalTok{Sum_Margin),}
             \DataTypeTok{colour =} \StringTok{'red'}\NormalTok{) }\OperatorTok{+}
\StringTok{  }\KeywordTok{geom_line}\NormalTok{(}\KeywordTok{aes}\NormalTok{(}\DataTypeTok{x =}\NormalTok{ Duration_table}\OperatorTok{$}\NormalTok{Duration_by_years, }\DataTypeTok{y =}\NormalTok{ Duration_table}\OperatorTok{$}\NormalTok{Sum_Margin),}
            \DataTypeTok{colour =} \StringTok{'blue'}\NormalTok{) }\OperatorTok{+}
\StringTok{  }\KeywordTok{ggtitle}\NormalTok{(}\StringTok{'Sum Margin vs Duration by years '}\NormalTok{) }\OperatorTok{+}
\StringTok{  }\KeywordTok{xlab}\NormalTok{(}\StringTok{'Duration by years'}\NormalTok{) }\OperatorTok{+}
\StringTok{  }\KeywordTok{ylab}\NormalTok{(}\StringTok{'Sum Margin'}\NormalTok{)}
\end{Highlighting}
\end{Shaded}

\includegraphics{Task_files/figure-latex/unnamed-chunk-26-1.pdf}

\begin{Shaded}
\begin{Highlighting}[]
\KeywordTok{ggplot}\NormalTok{() }\OperatorTok{+}
\StringTok{  }\KeywordTok{geom_point}\NormalTok{(}\KeywordTok{aes}\NormalTok{(}\DataTypeTok{x =}\NormalTok{ Duration_table}\OperatorTok{$}\NormalTok{Duration_by_years, }\DataTypeTok{y =}\NormalTok{ Duration_table}\OperatorTok{$}\NormalTok{percentage),}
             \DataTypeTok{colour =} \StringTok{'red'}\NormalTok{) }\OperatorTok{+}
\StringTok{  }\KeywordTok{geom_line}\NormalTok{(}\KeywordTok{aes}\NormalTok{(}\DataTypeTok{x =}\NormalTok{ Duration_table}\OperatorTok{$}\NormalTok{Duration_by_years, }\DataTypeTok{y =}\NormalTok{ Duration_table}\OperatorTok{$}\NormalTok{percentage),}
            \DataTypeTok{colour =} \StringTok{'blue'}\NormalTok{) }\OperatorTok{+}
\StringTok{  }\KeywordTok{ggtitle}\NormalTok{(}\StringTok{'Percentage of claims vs Duration by years'}\NormalTok{) }\OperatorTok{+}
\StringTok{  }\KeywordTok{xlab}\NormalTok{(}\StringTok{'Duration by years'}\NormalTok{) }\OperatorTok{+}
\StringTok{  }\KeywordTok{ylab}\NormalTok{(}\StringTok{'Percentage of claims'}\NormalTok{)}
\end{Highlighting}
\end{Shaded}

\includegraphics{Task_files/figure-latex/unnamed-chunk-26-2.pdf}

\begin{Shaded}
\begin{Highlighting}[]
\KeywordTok{ggplot}\NormalTok{() }\OperatorTok{+}
\StringTok{  }\KeywordTok{geom_point}\NormalTok{(}\KeywordTok{aes}\NormalTok{(}\DataTypeTok{x =}\NormalTok{ Duration_table}\OperatorTok{$}\NormalTok{Duration_by_years, }\DataTypeTok{y =}\NormalTok{ Duration_table}\OperatorTok{$}\NormalTok{Variance),}
             \DataTypeTok{colour =} \StringTok{'red'}\NormalTok{) }\OperatorTok{+}
\StringTok{  }\KeywordTok{geom_line}\NormalTok{(}\KeywordTok{aes}\NormalTok{(}\DataTypeTok{x =}\NormalTok{ Duration_table}\OperatorTok{$}\NormalTok{Duration_by_years, }\DataTypeTok{y =}\NormalTok{ Duration_table}\OperatorTok{$}\NormalTok{Variance),}
            \DataTypeTok{colour =} \StringTok{'blue'}\NormalTok{) }\OperatorTok{+}
\StringTok{  }\KeywordTok{ggtitle}\NormalTok{(}\StringTok{'Variance vs Duration by years'}\NormalTok{) }\OperatorTok{+}
\StringTok{  }\KeywordTok{xlab}\NormalTok{(}\StringTok{'Duration by years'}\NormalTok{) }\OperatorTok{+}
\StringTok{  }\KeywordTok{ylab}\NormalTok{(}\StringTok{'Variance'}\NormalTok{)}
\end{Highlighting}
\end{Shaded}

\includegraphics{Task_files/figure-latex/unnamed-chunk-26-3.pdf}

\subsubsection{Analysis conculsations}\label{analysis-conculsations-3}

Margin increase over time (as expected), variance decrease over time.
And number of complaints decrease after the 6th year of being insured.

\subsection{Step 4 - Test conclusions:}\label{step-4---test-conclusions}

\subsubsection{questions 1}\label{questions-1}

The most profitable costumer was mesured by the margin (=
profit/revenue). By using the multiple regression results and following
the charts I created we can see the following that the occupation
WHITE\_COLLAR has significant and positive impact on margin. In
addition, the WHITE\_COLLAR occupation's mean margin and the total
margin are positive and the variance is signnificantly small while the
two others occupations mean marin and total margin are negative and it's
variance is high. Therefore, the WHITE\_COLLAR customers are the most
profitable type of customer.

\subsubsection{questions 2}\label{questions-2}

The most risky customers were measured by their number of complaints.
The results show that people above the age of 50 are much more risky for
the company.

\subsubsection{questions 3}\label{questions-3}

I recommend for DataIns to focus on marketing on the following audience:

\begin{enumerate}
\def\labelenumi{\arabic{enumi}.}
\item
  There is a big potential of future profit by keeping existing
  customers that has been insured for more than 6 years since the amount
  of claims reduces over time. The fact that the insurance model is
  terminating the customer contract when a claim was submitted actually
  keeps the ``good'' customers. I would recommend to focus on existing
  customers that has been insured for more than 6 years and are younger
  than the age of 44.
\item
  New potential customers that has been insured for more than 6 years in
  a different company but younger than the age of 44 and have no record
  of complaint submitted.
\item
  Customers with the occupation of the type WHITE\_COLLAR.
\end{enumerate}


\end{document}
